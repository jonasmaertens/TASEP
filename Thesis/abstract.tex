\chapter*{Abstract}\label{chap:abstract}

Intelligent matter research is a relatively new, interdisciplinary field of study concerned with the description and development of novel materials that can adapt to external stimuli. Intelligent matter, a subset of active matter, is an example of physical systems that are far from equilibrium. The components that make up such systems draw energy from their surroundings to perform mechanical work. An interesting type of intelligent matter is created by combining synthetic matter with artificial intelligence. 
This study delves into the foundations of machine learning, particularly reinforcement learning with deep neural networks, and demonstrates how these algorithms can be integrated with the simple exclusion process to simulate systems of \textit{smarticles} (smart particles) and how this new type of matter can be used for transport optimization. 
Different examples of learned intelligent behavior are presented, and the effects of emergent global structure and social behavior on the total transport efficiency are discussed.
The research also sheds light on the broader implications of smart matter, positioning it as an intriguing research area in its own right. The thesis aims to serve as an early example of multi-agent smart matter simulations and offers the modular smarttasep package as an element for future research on similar topics.