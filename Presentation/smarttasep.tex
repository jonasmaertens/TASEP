\documentclass[10pt,xcolor=table, aspectratio=1610]{beamer}
% \special{dvipdfmx:config z 0}
\usetheme[progressbar=frametitle]{metropolis}
\usepackage[export]{adjustbox}
\usepackage{appendixnumberbeamer}
\usepackage{booktabs}
\usepackage{ellipsis}
\usepackage{hhline}
\usepackage{textgreek}
\usepackage{url}
\usepackage[outputdir=out]{minted}
\usepackage{csquotes}

\def\UrlBreaks{\do\/\do-}

\usepackage{caption}
\captionsetup{justification=raggedright,
              singlelinecheck=off,
              font=scriptsize,
              labelfont=scriptsize}
\setlength{\abovecaptionskip}{2pt}
\setlength{\belowcaptionskip}{0pt}

\setbeamertemplate{footline}{
  \begin{beamercolorbox}{footer nav}
    \vskip 2pt \insertnavigation{\paperwidth} \vskip 2pt
  \end{beamercolorbox}
}


\title{Smart TASEP}
\subtitle{Deep Q-Learning in Intelligent Matter Simulations}
\date{\today}
\author{Jonas Märtens}
\institute{LMU Munich}

\begin{document}
\maketitle

\section{Introduction}
\begin{frame}{Overview}
    \begin{columns}
      \column{0.5\textwidth}
      Smart TASEP: Deep Q-Learning in Intelligent Matter Simulations
      \begin{itemize}
        \item TASEP: Totally Asymmetric Simple Exclusion Process
        \item Smart TASEP: TASEP with intelligent agents \\
              $\rightarrow$ Intelligent Matter Simulations
        \item Deep Q-Learning: Reinforcement Learning with Neural Networks
      \end{itemize}
      \column{0.5\textwidth}
      cool TASEP teaser pic
    \end{columns}
\end{frame}


\begin{frame}{TASEP}
    \begin{columns}
      \column{0.5\textwidth}
      TASEP: Totally Asymmetric Simple Exclusion Process
      \begin{itemize}
        \item Stochastic Process on a grid lattice
        \item Exclusion: Only one particle per site
        \item Simple: Jumps only one cell far 
        \item Asymmetric: Only to the right
      \end{itemize}
      Possible with either
      \begin{itemize}
        \item Periodic Boundary Conditions
        \item Insertion and Extraction rates at the boundaries
      \end{itemize}
      \column{0.5\textwidth}
      both TASEP pics
    \end{columns}
\end{frame}

\begin{frame}{Modifications}
    \begin{columns}
      \column{0.5\textwidth}
      Two modifications (before smart)
      \begin{block}{(T)ASEP}
        \begin{itemize}
          \item Symmetric in one direction and totally asymmetric in the other
          \item Directed flow with multiple lanes
          \item compare traffic flow, kinesin transport
          \item motivates transport optimization
        \end{itemize}
      \end{block}
      \begin{block}{Speeds}
        \begin{itemize}
          \item Each particle has a speed
          \item Speed is probability to jump
          \item Drawn from a distribution
          \item If different, jams
        \end{itemize}
      \end{block}
      \column{0.5\textwidth}
      TASEP pic with speeds, maybe kinesin
    \end{columns}
\end{frame}

\section{Make it Smart!}
\begin{frame}{Smart TASEP}
    \begin{columns}
      \column{0.5\textwidth}
      Smart TASEP: TASEP with intelligent agents
      \begin{itemize}
        \item have a goal (e.g. go forward)
        \item sense the environment
        \item act according to the goal and the environment
        \item learn from the environment themselves
      \end{itemize}
      $\rightarrow$ Intelligent Matter Simulation \\
      $\rightarrow$ Reinforcement Learning - reward based
      \column{0.5\textwidth}
      Observation distance pic? Leading to action?
    \end{columns}
\end{frame}

\section{Deep Q-Learning}
\begin{frame}{Deep Q-Learning: Overview}
    \begin{columns}
      \column{0.5\textwidth}
      \begin{block}{Deep Q-Learning Algorithm}
        \begin{itemize}
          \item Reinforcement learning with deep neural networks
          \item Model-free: No knowledge of the environment
          \item Q-Learning: Value-based method
          \item Off-policy: Learns from old experiences
        \end{itemize}
      \end{block}
      \column{0.5\textwidth}
      DQL components
    \end{columns}
\end{frame}
\begin{frame}{Deep Q-Learning: The policy}
    large picture of flattened observation fed into a neural network, leading to a Q-value for each action
\end{frame}
\begin{frame}{Deep Q-Learning: Optimization}
    \begin{columns}
      \column{0.5\textwidth}
      \begin{block}{How does the network learn?}
        \begin{itemize}
          \item Neural Network $\rightarrow$ backpropagation on loss
          \item Problem: no target value for Q-Learning
          \item Solution: Use current best guess as target: \\
                $y = r + \gamma \max_{a'} \hat{Q}(s', a')$
          \item Target network with soft updates to stabilize learning
          \item Use gradients for AdamW optimizer
          \item Replay buffer
                \begin{itemize}
                  \item Sample efficiency
                  \item Break correlation
                \end{itemize}
        \end{itemize}
      \end{block}
      \column{0.5\textwidth}
      backpropagation pic
    \end{columns}
\end{frame}

\section{Implementation}
\begin{frame}{The \texttt{smarttasep} package}
    \begin{columns}
      \column{0.4\textwidth}
      \begin{block}{Features}
        \begin{itemize}
          \item Easy installation with \texttt{pip}
          \item Fully Customizable
          \begin{itemize}
            \item Environment size
            \item Reward function
            \item Network architecture
            \item Hyperparameters
            \item Training parameters
            \item Algorithm choices
          \end{itemize}
          \item Real-time visualization
          \item Interactive evaluation
          \item Management of experiments
        \end{itemize}
      \end{block}
      \column{0.6\textwidth}
      \begin{block}{\texttt{smarttasep} components}
        \begin{itemize}
          \item \texttt{smarttasep.GridEnv}: The 2D (T)ASEP environment
          \item \texttt{smarttasep.DQN}: The neural network
          \item \texttt{torchrl.ReplayBuffer}: Efficient, tensor-based replay buffer
          \item \texttt{smarttasep.Trainer}: Wrapper class for training, simulation and experiment management
          \item \texttt{smarttasep.Playground}: Real-time interactive evaluation of trained agents
          \item \texttt{smarttasep.EnvParams, smarttasep.Hyperparams}: Configuration classes
        \end{itemize}    
      \end{block}
    \end{columns}
\end{frame}

\begin{frame}{The \texttt{smarttasep} package: Examples}
Let's have a look at some examples! (show usage videos)
\end{frame}

\begin{frame}{Hyperparameter Optimization}
  For each experiment, the Hyperparameters should be optimized. 
    \begin{block}{Parameters}
      \begin{itemize}
        \item \textbf{Learning rate} of the AdamW optimizer.
        \item \textbf{Discount factor} $\gamma$.
        \item \textbf{Replay buffer size}.
        \item \textbf{Batch size} for training.
        \item \textbf{Target network update rate} $\tau$.
        \item \textbf{Exploration-exploitation tradeoff} via the decay constant $\eta$ of the exploration rate $\epsilon(t)=\epsilon_{\text{end}} + (\epsilon_{\text{start}} - \epsilon_{\text{end}}) e^{-n_{\text{steps}}/\eta}$.
        \item \textbf{Neural network architecture} (number of hidden layers and neurons per layer).
        \item \textbf{Activation function} of the neural network.
    \end{itemize}
    \end{block}
\end{frame}

\begin{frame}{Hyperparameter Optimization: Example}
    hidden layers sizes example
\end{frame}

\section[Baseline]{Results: Baseline}
\begin{frame}{Overview of Experiments}
  \begin{block}{Goals}
    \begin{itemize}
      \item Current optimization
      \item General insights into intelligent matter simulations  
    \end{itemize}
  \end{block}
  \begin{block}{Experiments}
  \begin{enumerate}
    \item \textbf{Baseline}: Classical 2D TASEP with speeds
    \item \textbf{Naive current optimization}: Hard-coded agents
    \item \textbf{Smart TASEP}: Deep Q-Learning agents with slightly different goals
    \begin{itemize}
      \item \textbf{Simple reward}: Go forward
      \item \textbf{Complex reward}: Go forward while forming lanes
    \end{itemize}
  \end{enumerate}
  \end{block}
\end{frame}

\begin{frame}{Results: Baseline}
  \begin{columns}
    \column{0.35\textwidth}
    \begin{block}{Setup}
      \begin{itemize}
        \item Classical 2D TASEP
        \item Periodic boundary conditions
        \item Checkerboard configuration
        \item<2-> Normally distributed speeds
        \item<3-> Waiting for steady state
        \item<4-> Measuring average current
      \end{itemize}
    \end{block}
    \column{0.6\textwidth}
    \begin{overprint}
    \onslide<2>\begin{figure}
      \includegraphics[width=0.65\textwidth]{../Thesis/img/results/truncated_normal.pdf}
      \caption*{\hspace{0.175\textwidth} Speeds of particles in the baseline experiment.}
    \end{figure}
    \onslide<3>\begin{figure}
      \includegraphics[width=\textwidth]{../Thesis/img/results/currents_fixed_sigma_128x32.pdf}
      \caption*{Equilibration of the current}
    \end{figure}
    \onslide<4->\begin{figure}
      \includegraphics[width=\textwidth]{../Thesis/img/results/steady_state_current_sizes_log.pdf}
      \caption*{Steady state current as a function of $\sigma$ for different system sizes.\\800 runs and 150k steps averaged.}
    \end{figure}
    \end{overprint}
  \end{columns}
\end{frame}

\begin{frame}{Results: Theory}
  \begin{columns}
    \column{0.35\textwidth}
    \begin{block}{Conditions for a forward move}
      \begin{enumerate}
        \item randomly picked cell is occupied ($p_\text{occ}=\rho$)
        \item speed allows move ($p_\text{spd}=\bar{v}=\mu$)
        \item forward direction is picked ($p=0.5$)
        \item next cell is empty ($p_\text{emp}=1-\rho$)$^*$
      \end{enumerate}
    \end{block}
    \begin{align*}
      \implies \left\langle J \right\rangle &= p \cdot \mu \cdot \rho \cdot (1-\rho) \\
      &= 0.0625
    \end{align*}
    \column{0.6\textwidth}
    \begin{figure}
      \includegraphics[width=\textwidth]{../Thesis/img/results/steady_state_current_sizes_log.pdf}
      \caption*{Steady state current as a function of $\sigma$ for different system sizes.}
    \end{figure}
  \end{columns}
\end{frame}

\section[Naive Policy]{Results: Naive Optimization}
\begin{frame}{A Naive Policy}
  \begin{columns}
    \column{0.35\textwidth}
    \begin{block}{Just go forward?}
      \begin{itemize}
        \item $p_\text{fwd} = 1, p_\text{up/down} = 0$
        \item starting in checkerboard configuration
        \item no messy mixing
        \item bad for high $\sigma$
        \item optimum for low $\sigma$?
      \end{itemize}
    \end{block}
    \begin{align*}
      \max \left\langle J \right\rangle &= \max \left( p_\text{occ} \cdot p_\text{spd} \cdot p_\text{fwd} \cdot p_\text{emp} \right) \\
      &\le \rho \cdot \mu \cdot 1 \cdot \max(p_\text{emp}) 
    \end{align*}
    \begin{equation*}
     \text{and}\quad 0.5 \le \max(p_\text{emp}) < 1
    \end{equation*}
    \column{0.6\textwidth}
    \begin{figure}
      \includegraphics[width=\textwidth]{../Thesis/img/results/steady_state_current_both_log.pdf}
      \caption*{Naive policy}
    \end{figure}
  \end{columns}
  
\end{frame}

\section[Smarticles]{Results: Smarticles}
\begin{frame}{Smarticles: Equal speeds}
  \begin{columns}
    \column{0.35\textwidth}
    \begin{block}{Simple reward}
      \begin{itemize}
        \item Positive reward for going forward
        \item Negative reward for occupied destination
        \item Zero reward for going up or down or staying
        \item Small negative reward for blocking others
      \end{itemize}
    \end{block}

    \column{0.65\textwidth}
      \begin{figure}
        \includegraphics[width=\textwidth]{../Thesis/img/results/first_training_screenshot.png}
        \caption*{Screenshot of the first smarticle training}
      \end{figure}
      
  \end{columns}

\end{frame}

\begin{frame}{Smarticles: Equal speeds}
  \begin{columns}
    \column{0.45\textwidth}
    \begin{block}{Resulting current}
      \begin{itemize}
        \item $\left\langle J \right\rangle \approx 0.152$
        \item $+143\%$ compared to baseline (0.0625)
        \item $+22\%$ compared to naive policy (0.125)
      \end{itemize}
    \end{block}
    \column{0.55\textwidth}
    \begin{figure}
        \includegraphics[width=\textwidth]{../Thesis/img/results/equal_speeds.pdf}
        \caption*{Steady state current over time for the trained smarticles.}
      \end{figure}
  \end{columns}
\end{frame}

\begin{frame}{Smarticles: Uniform speed distribution}
  \begin{columns}
    \column{0.35\textwidth}
    \begin{block}{Similar reward, $\sigma=10$}
      \begin{itemize}
        \item Same setup as before
        \item Negative reward for blocking proportional to speed
      \end{itemize}
    \end{block}
    \column{0.65\textwidth}
    \begin{figure}
        \includegraphics[width=\textwidth]{../Thesis/img/results/second_training_screenshot.png}
        \caption*{Screenshot of the second smarticle training}
      \end{figure}
  \end{columns}
\end{frame}

\begin{frame}{Smarticles: Uniform speed distribution}
  \begin{columns}
    \column{0.45\textwidth}
    \begin{block}{Resulting current}
      \begin{itemize}
        \item $\left\langle J \right\rangle \approx 0.117$
        \item $+125\%$ compared to baseline (0.052)
        \item Lower than with equal speeds (we will see why)
      \end{itemize}
    \end{block}
    \column{0.55\textwidth}
    \begin{figure}
        \includegraphics[width=\textwidth]{../Thesis/img/results/uniform_speeds.pdf}
        \caption*{Steady state current over time for the trained smarticles.}
      \end{figure}
  \end{columns}
\end{frame}

\begin{frame}{Smarticles: Model Analysis}
  What did the smarticles learn? (Playground video)
\end{frame}

\begin{frame}{Smarticles: Why Global Structures}
  \begin{columns}
    \column{0.5\textwidth}
    \begin{block}{Observation}
      \begin{itemize}
        \item So far, the system still looks messy
        \item Only individual behavior
        \item Compare highway: Fast lane and slow lane
        $\rightarrow$ Could be beneficial also in TASEP 
      \end{itemize}
    \end{block}
    \begin{block}{Potential}
      \begin{itemize}
        \item Less jamming
        \item Directed flow without dispersion
        \item Inhomogeneous density
      \end{itemize}
    \end{block}
    \column{0.5\textwidth}
    messy tasep pic
    fast lane slow lane picture
  \end{columns}
\end{frame}

\begin{frame}{Smarticles: Hard-coded Lanes}
  Approach 1: Hard code the lanes, supply smarticles with lane information\\
  \begin{equation*}
    \Delta y = 1-a\cdot\left(\frac{1+a}{a}\right)^v+a
  \end{equation*}
  \begin{figure}
    \includegraphics[width=\textwidth]{../Thesis/img/results/speed_gradient_0.4.pdf}
    \caption*{Learned smarticle behavior with the hard-coded lanes reward function (left), and the speed gradient mapping (right).}
  \end{figure}
\end{frame}

\begin{frame}{Smarticles: Hard-coded Lanes - Results}
  \begin{columns}
    \column{0.45\textwidth}
    \begin{block}{Resulting current}
      \begin{itemize}
        \item $\left\langle J \right\rangle \approx 0.134$
        \item $+157\%$ compared to baseline (0.052)
        \item $+15\%$ compared to previous policy (0.117)
        \item long equilibration time
      \end{itemize}
    \end{block}
    \column{0.55\textwidth}
    \begin{figure}
        \includegraphics[width=\textwidth]{../Thesis/img/results/speed_grad_current.pdf}
        \caption*{Current over time for the trained smarticles.}
      \end{figure}
  \end{columns}
\end{frame}

\begin{frame}{Smarticles: Self-organized Lanes}
  Approach 2: Let the smarticles learn the lanes themselves from local \enquote{interactions}\\
  \begin{equation*}
    \text{V}(r, \Delta v) = \begin{cases}
      -0.125 \cdot r + 0.625 & \text{if } \Delta v < 0.5 \text{ and } 1.5 < r \le 5 \\
      -0.75 \cdot r^{-1.3} - 0.15 & \text{if } \Delta v \ge 0.5 \text{ and } r \le 3.5 \\
      0 & \text{otherwise}
  \end{cases}
  \end{equation*}
  \begin{block}{Modifications}
    \begin{enumerate}
      \item Binary speed distribution
      \item Different neural networks
    \end{enumerate}
  \end{block}
\end{frame}

\begin{frame}
  \begin{figure}
    \includegraphics[width=0.6\textwidth]{../Thesis/img/results/lane_reward_func_3d_cropped.pdf}
    \caption*{\hspace{0.2\textwidth}Visual representation of the lane reward function / \enquote{potential}}
  \end{figure}
\end{frame}

\begin{frame}{Smarticles: Self-organized Lanes - Results}
  Let's see it in action! (lane formation video)
\end{frame}

\begin{frame}{Smarticles: Self-organized Lanes - Results}
  \begin{columns}
    \column{0.45\textwidth}
    \begin{block}{Resulting current}
      \begin{itemize}
        \item $\left\langle J \right\rangle \approx 0.112$
        \item baseline $J=0.4\cdot(1-0.4)\cdot0.6\cdot0.5=0.072$
        \item $+56\%$ compared to baseline
        \item lower than previous policies
        \item quicker equilibration
        \item fluctuations seem higher than they are
        \item proof of concept, room for improvement
      \end{itemize}
    \end{block}
    \column{0.55\textwidth}
    \begin{figure}
        \includegraphics[width=\textwidth]{../Thesis/img/results/lanes_current.pdf}
        \caption*{Current over time for the trained smarticles.}
      \end{figure}
  \end{columns}
\end{frame}

\section{Conclusion and Outlook}
\begin{frame}{Conclusion and Outlook}

\end{frame}



\end{document}